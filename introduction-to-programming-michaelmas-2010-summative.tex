% introduction-to-programming-michaelmas-2010-summative.tex -- Summative report
%
% This is a solution of the summative assignment of the Introduction to
% Networks submodule of the Introduction to Programming module held at the
% Durham University, Durham, United Kingdom.  Michælmas term 2010.
%
% Based on \cite{minar}
%
% Copyright © 2008–2010 Jan Minář <rdancer@rdancer.org>
%
% This work is free software; you can redistribute it and/or modify
% it under the terms of the GNU General Public License version 2 (two),
% as published by the Free Software Foundation.
%
% This work is distributed in the hope that it will be useful,
% but WITHOUT ANY WARRANTY; without even the implied warranty of
% MERCHANTABILITY or FITNESS FOR A PARTICULAR PURPOSE.  See the
% GNU General Public License for more details.
%
% You should have received a copy of the GNU General Public License along
% with this work; if not, write to the Free Software Foundation, Inc.,
% 51 Franklin Street, Fifth Floor, Boston, MA 02110-1301 USA.

\documentclass[10pt]{report}


\usepackage[utf8]{inputenc}
\pdfoutput=1
\usepackage[pdftex]{graphicx}     
\usepackage{amssymb}    
\usepackage{harvard}
\usepackage{url}
\usepackage{fancyhdr}
\usepackage{lastpage}

\pagestyle{fancy}
\fancyhead{}
%\chead{Introduction to Programming --- Summative Assignment}
%\chead{Copyright © 2008–2010 Jan Minář {\tt <rdancer@rdancer.org>}}
	%\\ page \thepage\ of \pageref{LastPage}}

\chead{
    Introduction to Programming --- Summative Assignment\\
    Copyright © 2008–2010 Jan Minář {\tt <rdancer@rdancer.org>}
}

\author{Jan Minář {\tt <rdancer@rdancer.org>}}
%\date{November 28, 2008}

%%%%%%%%%%%%%%%%%%%%%%%%%%%%%%%%%%%%%%%%%%%%%%%%%%%%%%%%%%%%%%%%%%%%%%%%%%%%%%
% ``Title''
%	-- the assignment

% No LaTeX command to make a subtitle, but possible using custom code (not
% including due to absence of license, but it is possible):
% <http://groups.google.com/group/comp.text.tex/msg/3aa4f67d8b3a979b?hl=en-EN&pli=1>
\title{Introduction to Programming\\Summative Assignment}

\begin{document}
\bibliographystyle{agsm}

\maketitle

% Note: This is a ‘report’


%%%%%%%%%%%%%%%%%%%%%%%%%%%%%%%%%%%%%%%%%%%%%%%%%%%%%%%%%%%%%%%%%%%%%%%%%%%%%%
% ``Abstract (10%) Brief (about 100 words) summary of the main points. Write
% this last.''
%	-- the assignment
\chapter{Abstract}
\thispagestyle{fancy}
This work is in part based on \cite{minar}

%%%%%%%%%%%%%%%%%%%%%%%%%%%%%%%%%%%%%%%%%%%%%%%%%%%%%%%%%%%%%%%%%%%%%%%%%%%%%%
% ``Introduction (20%) Explain the hypothesis(es) of the experiment so an
% informed reader can understand why you are doing what you are doing.''
%	-- the assignment
\chapter{Introduction}
\thispagestyle{fancy}

%%%%%%%%%%%%%%%%%%%%%%%%%%%%%%%%%%%%%%%%%%%%%%%%%%%%%%%%%%%%%%%%%%%%%%%%%%%%%%
% ``Method (20%) What did you did, in enough detail that somebody else could
% replicate the experiment if they so wished. Feel free to adapt my
% instructions. Make explicit any assumptions or limitations of the method.''
%	-- the assignment
\chapter{Method}
\thispagestyle{fancy}

We are going to be using OpenJDK, Java version 1.6.
\begin{verbatim}
$ java -version
java version "1.6.0_18"
OpenJDK Runtime Environment (IcedTea6 1.8.2) (6b18-1.8.2-4ubuntu2)
OpenJDK 64-Bit Server VM (build 16.0-b13, mixed mode)
\end{verbatim}


\section{Runtime Testing}

The program is rather large, and there is no time to develop a full test suite.  We will only have to test 30 data points, and therefore we will employ white box testing, where we will ensure to run through the affected source code statement(s).  No testing will be performed if the source code change does not effect a semantic change.

%%%%%%%%%%%%%%%%%%%%%%%%%%%%%%%%%%%%%%%%%%%%%%%%%%%%%%%%%%%%%%%%%%%%%%%%%%%%%%
% ``Results (20%) Present your results numerically, using tables, graphs,
% charts, summary statistics (e.g. averages) as appropriate.''
%	-- the assignment
\chapter{Results}
\thispagestyle{fancy}

\begin{figure}[p]
    \centering
    % Border around the image
    \setlength\fboxsep{0pt}
    \setlength\fboxrule{0.5pt}
    \fbox{
	\includegraphics[width=0.8\textwidth]{placeholder-pie-chart.png}
    }
    \caption{Fault Injection Effect}
    \label{placeholder}
\end{figure}

Figure \ref{placeholder} shows XXX XXX.

\begin{figure}[p]
    \centering
    % Border around the image
    \setlength\fboxsep{0pt}
    \setlength\fboxrule{0.5pt}
    \fbox{
	\includegraphics[width=0.8\textwidth]{placeholder-pie-chart-error-identification.png}
    }
    \caption{Error Identification}
    \label{placeholder-error-identification}
\end{figure}

Figure \ref{placeholder-error-identification} shows XXX XXX.



%%%%%%%%%%%%%%%%%%%%%%%%%%%%%%%%%%%%%%%%%%%%%%%%%%%%%%%%%%%%%%%%%%%%%%%%%%%%%%
% ``Discussion (20%) What do your results have to say about the hypotheses?''
%	-- the assignment
\chapter{Discussion}
\thispagestyle{fancy}



%%%%%%%% \chapter{Client-Server Architecture}
%%%%%%%% \thispagestyle{fancy}
%%%%%%%% 
%%%%%%%% In a client-server architecture, the application is distributed over two or
%%%%%%%% more separate platforms.  The servers offer services which are utilized by the
%%%%%%%% clients.  One functional unit can act both as a client and a server at the same time
%%%%%%%% (e.g.\ a web server that is at the same time a client to a database server).
%%%%%%%% Clients and servers communicate over shared network. \cite[pp3--11]{vaughn}
%%%%%%%% 
%%%%%%%% There is a vast number of applications that follow the client-server model
%%%%%%%% (most of the ports assigned by IANA correspond to client-server applications
%%%%%%%% \cite{iana}).  Alternatives to the client-server architecture include
%%%%%%%% {\em application} architecture and {\em peer-to-peer} architecture.  \cite[p110]{kurose}
%%%%%%%% 
%%%%%%%% There is typically one or a few servers, serving a large number of clients
%%%%%%%% \cite[p110]{kurose}.  It is not possible for clients to communicate
%%%%%%%% with each other directly; all communication between clients must be
%%%%%%%% realized through the server (for example, two Mail User Agents 
%%%%%%%% can indeed send e-mails to each other, but always via e.g.\ SMTP and IMAP mail servers).
%%%%%%%% 
%%%%%%%% Maintaining a server can be ``infrastructure-intensive'', when the server has
%%%%%%%% to withstand very many requests.  Often a server farm is deployed, so that the
%%%%%%%% load is shared between multiple machines.  \cite[p110]{kurose}
%%%%%%%% 
%%%%%%%% \section{Transport Layer View}
%%%%%%%% 
%%%%%%%% The client initiates the communication by sending a request to the
%%%%%%%% server.  The server only {\em responds}, and can never {\em initiate}
%%%%%%%% communication.  If the initial request is to succeed, the server process
%%%%%%%% must have had requested the operating system to listen on a given (TCP
%%%%%%%% or UDP) port.  The port numbers and their corresponding services are
%%%%%%%% maintained by a central registry \cite{iana}, so that for example a POP3
%%%%%%%% client will be able to connect to a POP3 server just by knowing the
%%%%%%%% server IP address.  The port number only has to be specified when it differs
%%%%%%%% from the default.  Once the client sends in a request to the correct port and
%%%%%%%% address, request is received by the operating system of the server machine, the
%%%%%%%% operating system passes the request on to the server process.  The server
%%%%%%%% process responds to the request, and two-way communication ensues.
%%%%%%%% 
%%%%%%%% % % % % % % % % % % % % % % % % % % % % % % % % % % % % % % % % % % % % % % %
%%%%%%%% 
%%%%%%%% \section{Thick and Thin Clients}
%%%%%%%% 
%%%%%%%% A thin client system does most of the data processing on the server side.  This
%%%%%%%% allows the client to be relatively low-powered, which can mean lower costs
%%%%%%%% (network terminals used instead of full-blown PCs, with the applications running
%%%%%%%% on an application server), or perhaps longer battery life (mobile phone or a PDA
%%%%%%%% running a video-processing application client, with the actual computationally
%%%%%%%% intensive video re-encoding performed on the server accessed over the mobile network).  Early web browsers were thin clients.
%%%%%%%% 
%%%%%%%% A thick client does most of the data processing itself.  This approach
%%%%%%%% does not suffer from the limitations of the network, such as latency.
%%%%%%%% For example, high quality video playback is a bandwidth-intensive
%%%%%%%% application, and it is often more practical to transport the video over
%%%%%%%% the network in a compressed form, and have the client do the
%%%%%%%% computationally intensive decompression, thus saving bandwidth.
%%%%%%%% Contemporary web browsers are thick clients.
%%%%%%%% 
%%%%%%%% %%%%%%%%%%%%%%%%%%%%%%%%%%%%%%%%%%%%%%%%%%%%%%%%%%%%%%%%%%%%%%%%%%%%%%%%%%%%%%
%%%%%%%% 
%%%%%%%% %%%%%%%%%%%%%%%%%%%%%%%%%%%%%%%%%%%%%%%%%%%%%%%%%%%%%%%%%%%%%%%%%%%%%%%%%%%%%%
%%%%%%%% % ``There are TWO types of packet switching networks: virtual circuit and
%%%%%%%% % datagram.  Describe in detail the key features of each type?''
%%%%%%%% %	-- the assignment
%%%%%%%% 
%%%%%%%% \chapter{Key Features of Virtual Circuit and Datagram Packet Switching Networks}
%%%%%%%% \thispagestyle{fancy}
%%%%%%%% 
%%%%%%%% % Intro
%%%%%%%% 
%%%%%%%% {\em Virtual circuit} (VC; also called {\em virtual call}) networks came from the
%%%%%%%% traditional pre-existing voice telephone network, which was usually a state-wide
%%%%%%%% network, with interstate and overseas links.  The word ``datagram'' itself
%%%%%%%% derives from ``telegram'' \cite[p141]{russell}.  On the other hand, {\em packet
%%%%%%%% switching} was developed in the 1970s as means of efficient transmission of data
%%%%%%%% over long distances.  Nowadays, datagram networking takes over traditional mainstays of VC,
%%%%%%%% however, VC is still being used.  It is possible to use a mixed approach.
%%%%%%%% 
%%%%%%%% \section{Virtual Circuit Lifetime}
%%%%%%%% 
%%%%%%%% Connection via VC has three phases:
%%%%%%%% 
%%%%%%%% \begin{enumerate}
%%%%%%%% \item set-up
%%%%%%%%     \begin{itemize}
%%%%%%%%     %\item LCI (Logical Channel Identifier; the name of the single connection to the next node) is chosen
%%%%%%%%     \item the path through the network is determined
%%%%%%%%     \item resources such as bandwidth/time slot are reserved, and everything is set
%%%%%%%%     \item this can take some amount of time, creating a set-up delay
%%%%%%%%     \end{itemize}
%%%%%%%% \item data transfer
%%%%%%%%     \begin{itemize}
%%%%%%%%     %\item nodes use LCI
%%%%%%%%     \item the path does not changed for the duration of the connection
%%%%%%%%     \item resources remain reserved even when not actually needed
%%%%%%%%     \item the latency is negligible, because there are no delays introduced by the management of the link, unlike with datagram networks
%%%%%%%%     \end{itemize}
%%%%%%%% \item tear-down
%%%%%%%%     \begin{itemize}
%%%%%%%%         %\item the resources are freed
%%%%%%%% 	%\item LCI is forgotten
%%%%%%%% 	\item the routing table entries are purged
%%%%%%%% 	\item bandwidth/slots are released for use for future VCs
%%%%%%%%     \end{itemize}
%%%%%%%% \end{enumerate}
%%%%%%%% 
%%%%%%%% \section{Routing in Datagram Networks}
%%%%%%%% 
%%%%%%%% Datagram networks route each packet individually.
%%%%%%%% When a datagram router receives a packet, it needs to decide which next
%%%%%%%% hop it should send it to, i.e.\ which interface to forward it via.  It
%%%%%%%% would be impractical to store this information for every possible
%%%%%%%% destination address, and therefore the routers mostly store only
%%%%%%%% aggregate routes (multiple adjacent addresses represented by a common
%%%%%%%% prefix).  Often an address is
%%%%%%%% comprised of a prefix, and a host part.  The routes are then
%%%%%%%% decided with respect to the network prefixes, not the individual
%%%%%%%% addresses.  A router can maintain two different routes for two network
%%%%%%%% prefixes in such a way that one prefix is contained in the other one.
%%%%%%%% In that case, the longer prefix takes precedence.  It is said that the
%%%%%%%% corresponding route is more specific.
%%%%%%%% 
%%%%%%%% As using hard-coded, or static routing would not be feasible for busy
%%%%%%%% routers with many connections, automatic routing protocols have been
%%%%%%%% devised that change routing table typically every few minutes.  In
%%%%%%%% contrast to that, in a VC network, the routing table constantly changes
%%%%%%%% with every VC setup/tear-down, many times a second.
%%%%%%%% 
%%%%%%%% 
%%%%%%%% %\section{History}
%%%%%%%% %
%%%%%%%% %In telephony systems, complexity is kept within the network, which allows
%%%%%%%% %``dumb'' end-systems as simple as rotary phones be connected.  Internet was on
%%%%%%%% %the other hand designed to connect sophisticated hosts, and designed
%%%%%%%% %deliberately to be ``as simple as possible''.  Complex functionality is
%%%%%%%% %implemented by the {\em transport} and {\em application} layers.  The internet
%%%%%%%% %{\em network} layer is easy to implement over disparate range of {\em link} layer
%%%%%%%% %technologies, such as ``satellite, Ethernet, fiber, or radio''.  It's also easy
%%%%%%%% %to implement new applications, because all that is required is to implement them
%%%%%%%% %in the end hosts; changes to the network are not necessary.
%%%%%%%% %\cite[pp349--351]{kurose}
%%%%%%%% 
%%%%%%%% \section{Comparison}
%%%%%%%% 
%%%%%%%% \begin{table}[h]
%%%%%%%%   \begin{tabular}{ | p{6cm} | p{6cm} | }
%%%%%%%%   \hline
%%%%%%%% 
%%%%%%%%   {\em Virtual Circuit}
%%%%%%%%   	& {\em Datagram} \\ \hline
%%%%%%%%   \hline
%%%%%%%% 
%%%%%%%% complexity is kept within the network, which allows end-systems to be simple,
%%%%%%%% even ``dumb'' \cite[p349]{kurose}
%%%%%%%% 	& designed to connect sophisticated
%%%%%%%% 	hosts, the network is deliberately ``as simple as possible'';  complex functionality is implemented by the {\em transport} and {\em application} layers \cite[pp349--351]{kurose}
%%%%%%%% 	\\ \hline
%%%%%%%% 
%%%%%%%% maintains a dedicated path through the network between the two end-hosts 
%%%%%%%% 	& routes each packet separately, each packet contains routing
%%%%%%%% 	information header; path can change in-between packets \\ \hline
%%%%%%%% 
%%%%%%%% guarantees delivery, low latency and reserved bandwidth
%%%%%%%% 	& best-effort, with packet loss/duplicity, jitter, data corruption,
%%%%%%%% 	shared bandwidth \\ \hline
%%%%%%%% 
%%%%%%%% its all-or-nothing approach means it will simply fail hard and tear down the
%%%%%%%% connection in case of an error
%%%%%%%% 	&  routes around a malfunctioning router or network path and recovers
%%%%%%%% 	from errors \\ \hline
%%%%%%%% 
%%%%%%%% naturally maps onto traditional voice networks, which it was developed from and
%%%%%%%% for
%%%%%%%% 	& developed in the 1970s; has not fundamentally changed since, one
%%%%%%%% 	of few technologies of choice for data transmission over long distances
%%%%%%%% 	\cite[p298--299]{stallings} \\ \hline
%%%%%%%% 
%%%%%%%% good for voice, video \& similar real-time, low-latency applications that
%%%%%%%% require approximately the same amount of bandwidth all the time
%%%%%%%% 	& good for applications that transmit data in bursts \\ \hline
%%%%%%%% 
%%%%%%%% upfront delay while the virtual circuit is set up; latency negligible afterwards
%%%%%%%% 	& considerable latency due to intrinsic delays \\ \hline
%%%%%%%% 
%%%%%%%%   \end{tabular}
%%%%%%%%   \caption{Comparison of Virtual Circuit and Datagram Networks}
%%%%%%%%   \label{vcdgcomparison}
%%%%%%%% \end{table} 
%%%%%%%% 
%%%%%%%% 
%%%%%%%% Table \ref{vcdgcomparison} compiled from data in \cite{kurose,russell},
%%%%%%%% \cite[p298--299]{stallings}.
%%%%%%%% 
%%%%%%%% %Some see it as an advantage that instead of using the (long) address, only the
%%%%%%%% %(short) Logical Channel Identifier (LCI) is used.  This will save bandwidth,
%%%%%%%% %complexity and processing power, compared to having to compute the route for
%%%%%%%% %every datagram. \cite[p158]{russell} (the addresses are only used/needed during
%%%%%%%% %the initial phase of connection set-up).  However, there are ways to approximate
%%%%%%%% %these advantages even in datagram networks.
%%%%%%%% %
%%%%%%%% %Virtues of the VC technology do not prevent errors in other parts of
%%%%%%%% %the application.  Properly engineered system will function correctly
%%%%%%%% %regardless of which of the two approaches is used \cite[p161]{russell}
%%%%%%%% %
%%%%%%%% %Some applications can not take advantage of the VC guaranteed
%%%%%%%% %sequentiality \cite[p161]{russell}
%%%%%%%% %
%%%%%%%% %The VC advantage of link and CPU efficiency due to using the LCI in
%%%%%%%% %place of the full address is not so marked when workarounds and
%%%%%%%% %real-life costs are taken into account: datagram network does not have
%%%%%%%% %to work with the full-length representation of an address, and replacing
%%%%%%%% %the (long) address with a (short) LCI does not necessarily prove to
%%%%%%%% %result in significant savings \cite[p161]{russell}
%%%%%%%% 
%%%%%%%% \section{Hybrid Approach}
%%%%%%%% 
%%%%%%%% ``[V]irtual circuit can be built on top of the datagram service''
%%%%%%%% \cite[p141]{russell}.  Many currently deployed networks behave like a VC towards the end-hosts,
%%%%%%%% and are really internally working as datagram networks---the advantage
%%%%%%%% of having a comfortable interface presented to the host is married with
%%%%%%%% the flexibility and cost-efficiency of a datagram network.
%%%%%%%% %%%%%%%%%%%%%%%%%%%%%%%%%%%%%%%%%%%%%%%%%%%%%%%%%%%%%%%%%%%%%%%%%%%%%%%%%%%%%%
%%%%%%%% 
%%%%%%%% 
%%%%%%%% 
%%%%%%%% %%%%%%%%%%%%%%%%%%%%%%%%%%%%%%%%%%%%%%%%%%%%%%%%%%%%%%%%%%%%%%%%%%%%%%%%%%%%%%
%%%%%%%% % ``There are a wide range of application layer protocols including the
%%%%%%%% % Hypertext Transfer Protocol (HTTP).  Describe HTTP in detail and be sure to
%%%%%%%% % include in your description HTTP's function (what does it do), design (why
%%%%%%%% % does it do it), and behaviour (how does it do it).''
%%%%%%%% %	-- the assignment
%%%%%%%% 
%%%%%%%% \chapter{Hypertext Transfer Protocol}
%%%%%%%% \thispagestyle{fancy}
%%%%%%%% %%%%%%%%%%%%%%%%%%%%%%%%%%%%%%%%%%%%%%%%%%%%%%%%%%%%%%%%%%%%%%%%%%%%%%%%%%%%%%
%%%%%%%% 
%%%%%%%% The Hypertext Transfer Protocol is a public domain client-server
%%%%%%%% stateless application layer protocol that communicates over TCP.
%%%%%%%% \cite{rfc1945,rfc2616},
%%%%%%%% \cite[pp122--124]{kurose}
%%%%%%%% 
%%%%%%%% HTTP is the most often used data transport application protocol on the
%%%%%%%% World Wide Web.
%%%%%%%% 
%%%%%%%% \section{Function}
%%%%%%%% 
%%%%%%%% HTTP retrieves and manipulates objects denoted by Uniform Resource Identifiers
%%%%%%%% (URIs).  \cite{stallings} HTTP methods GET, PUT, DELETE, MOVE, LINK, UNLINK are
%%%%%%%% used for this purpose.  HTTP has auxiliary capabilities represented by the
%%%%%%%% methods HEAD OPTIONS, TRACE, CONNECT and others.
%%%%%%%% 
%%%%%%%% HTTP has built-in support for caches.  Cache is an intermediate web server that
%%%%%%%% takes the HTTP requests and if possible serves the objects from a local
%%%%%%%% repository, thereby providing speedier response and saving Internet bandwidth.
%%%%%%%% 
%%%%%%%% \section{Behaviour}
%%%%%%%% 
%%%%%%%% HTTP uses several powerful technologies, some of them predating its invention,
%%%%%%%% some of them invented because of HTTP.  Underlying character set is
%%%%%%%% human-readable ASCII.  Reliable transport is provided by TCP.  The protocol
%%%%%%%% leverages the very powerful concept of URIs/URLs \cite{rfc1738} to identify the
%%%%%%%% objects that are to be manipulated.  HTTP also uses MIME, and other technologies
%%%%%%%% to negotiate the presentation of the object identified by the URI.\cite{rfc2616}
%%%%%%%% HTTP then retrieves the object.
%%%%%%%% 
%%%%%%%% \section{Design}
%%%%%%%% 
%%%%%%%% HTTP was designed to transfer simple textual web pages on the World Wide Web
%%%%%%%% efficiently.  Because a typical web session has the user retrieving pages in a
%%%%%%%% quick succession from different servers, the protocol was designed to have low
%%%%%%%% overhead, and be stateless.  By the time HTTP 1.1 was designed, the Web changed
%%%%%%%% radically, and a need for multi-object web pages was felt, as well as usefulness
%%%%%%%% of tracking the user across visits.  Therefore, persistent connections and
%%%%%%%% pipelining was added, and HTTP cookies.  Other functionality, such as the
%%%%%%%% support for virtual servers (the Host header) has been added.
%%%%%%%% 
%%%%%%%% 
%%%%%%%% \section{Packet Dissection}
%%%%%%%% 
%%%%%%%% We have prepared two HTTP packets out of a TCP stream that was the
%%%%%%%% result of retrieving a single document from a remote WWW server.  The
%%%%%%%% upper half of the images shows the interpretation, the lower half then
%%%%%%%% the on-the-wire data in hexadecimal and ASCII representation.  We used
%%%%%%%% {\em Wireshark} 1.0.0 to perform the packet analysis.
%%%%%%%% 
%%%%%%%% \begin{figure}[p]
%%%%%%%%     \centering
%%%%%%%%     % Border around the image
%%%%%%%%     \setlength\fboxsep{0pt}
%%%%%%%%     \setlength\fboxrule{0.5pt}
%%%%%%%%     \fbox{
%%%%%%%% 	\includegraphics[width=0.8\textwidth]{http-get.png}
%%%%%%%%     }
%%%%%%%%     \caption{HTTP request}
%%%%%%%%     \label{httprequest}
%%%%%%%% \end{figure}
%%%%%%%% 
%%%%%%%% Figure \ref{httprequest} shows the HTTP request. 
%%%%%%%% TCP packet sent to HTTP default port (denoted by {\em http}; 80), the well-known port number assigned to HTTP.
%%%%%%%% The request line specifies the GET method, path /, the HTTP version used is 1.0.
%%%%%%%% The User Agent (UA) claims to be Wget version 1.10.2, and it accepts any and all MIME types.
%%%%%%%% URI host part is example.com.  The UA wishes to use persistent connections.
%%%%%%%% Entity body is empty.
%%%%%%%% 
%%%%%%%% \begin{figure}[p]
%%%%%%%%     \centering
%%%%%%%%     % Border around the image
%%%%%%%%     \setlength\fboxsep{0pt}
%%%%%%%%     \setlength\fboxrule{0.5pt}
%%%%%%%%     \fbox{
%%%%%%%% 	\includegraphics[width=0.8\textwidth]{http-200-ok.png}
%%%%%%%%     }
%%%%%%%%     \caption{HTTP response}
%%%%%%%%     \label{httpresponse}
%%%%%%%% \end{figure}
%%%%%%%% 
%%%%%%%% Figure \ref{httpresponse} shows the HTTP response.
%%%%%%%% The status line shows the server uses HTTP version 1.1, status code 200 means
%%%%%%%% that the request has been successfully processed.  The response was generated on
%%%%%%%% 2009-01-30 at 06:54:00 GMT.  Server claims to be Apache version 2.2.3 running on
%%%%%%%% CentOS.  The requested object has not been modified since 2005-11-15 13:24:10
%%%%%%%% GMT.  The entity tag (ETag) is given.  Server accepts byte-range requests.
%%%%%%%% Length of the entity body is 438 octets.  Server will close the TCP connection
%%%%%%%% when the entity body will have been transmitted.  The MIME type of the object is
%%%%%%%% text/html, character set UTF-8.  The entity body contains the requested object
%%%%%%%% (the HTML page).
%%%%%%%% 
%%%%%%%% %%%%%%%%%%%%%%%%%%%%%%%%%%%%%%%%%%%%%%%%%%%%%%%%%%%%%%%%%%%%%%%%%%%%%%%%%%%%%%
%%%%%%%% % ``There are TWO types of transport layer service that the Internet provides
%%%%%%%% % to its applications, connection-oriented services and connectionless
%%%%%%%% % services.
%%%%%%%% % ``a) Describe in detail the characteristics of connection-oriented services.
%%%%%%%% % ``b) Describe in detail the characteristics of connectionless services.''
%%%%%%%% %	-- the assignment
%%%%%%%% 
%%%%%%%% \chapter{Characteristics of Connection-Oriented and Connectionless Transport
%%%%%%%% 	Layer Services}
%%%%%%%% \thispagestyle{fancy}
%%%%%%%% 
%%%%%%%% The two most prolific Internet Protocol Suite transport layer protocols are UDP
%%%%%%%% and TCP.  We will show the characteristics of connection-oriented and
%%%%%%%% connectionless services using UDP and TCP as examples and describing briefly how
%%%%%%%% they work.
%%%%%%%% 
%%%%%%%% \section{Connectionless Service---User Datagram Protocol}
%%%%%%%% 
%%%%%%%% UDP is defined in \cite{rfc768}.  It is the datagram transport in the
%%%%%%%% Internet Protocol Suite.  It is a very thin layer on top
%%%%%%%% of IP, as it only provides a checksum and port number (enabling
%%%%%%%% multiplexing, i.e.\ having more than one connection from one host to
%%%%%%%% another via UDP). (Cf.\ Fig.\ \ref{udpheader}) \cite{rfc1180}
%%%%%%%% 
%%%%%%%% \begin{figure}[h]
%%%%%%%%     \label{udpheader}
%%%%%%%%     \centering
%%%%%%%%     % Border around the image
%%%%%%%% 	\begin{verbatim}
%%%%%%%%               0      7 8     15 16    23 24    31  
%%%%%%%%               +--------+--------+--------+--------+
%%%%%%%%               |     Source      |   Destination   |
%%%%%%%%               |      Port       |      Port       |
%%%%%%%%               +--------+--------+--------+--------+
%%%%%%%%               |                 |                 |
%%%%%%%%               |     Length      |    Checksum     |
%%%%%%%%               +--------+--------+--------+--------+
%%%%%%%%               |                                    
%%%%%%%%               |          data octets ...           
%%%%%%%%               +---------------- ...                
%%%%%%%% 	\end{verbatim}
%%%%%%%%     \caption{User Datagram Protocol Header
%%%%%%%% 	    % XXX DNW
%%%%%%%% 	    %\cite{rfc768}
%%%%%%%% 	    (Postel 1980, p1) % Verbatim: DWIT!
%%%%%%%%     }
%%%%%%%% \end{figure}
%%%%%%%% 
%%%%%%%% The connectionless service only provides the bare minimum necessary for
%%%%%%%% a transport layer to provide.  This can be a positive feature, because
%%%%%%%% it means low overhead and low implementation complexity.
%%%%%%%% 
%%%%%%%% UDP provides the application layer with the ability to send and receive short
%%%%%%%% fixed-length datagrams.   Each datagram is a separate transmission; there is no
%%%%%%%% state kept in-between.  If desired, the application can implement some of the
%%%%%%%% missing features in a higher layer.
%%%%%%%% 
%%%%%%%% \section{Connection-Oriented Service---Transport Control Protocol}
%%%%%%%% 
%%%%%%%% TCP \cite{rfc793} is the most commonly used transport protocol in the
%%%%%%%% Internet Protocol Suite.  In order to provide a connection-oriented service, the
%%%%%%%% protocol is necessarily more complex than its connectionless
%%%%%%%% counterpart.  The complexity comes at cost in bandwidth and processing
%%%%%%%% overhead.
%%%%%%%% 
%%%%%%%% TCP provides in-order reliable delivery (with automatic retransmission
%%%%%%%% of lost packets) and automatic congestion sensing and adaptation.  Like
%%%%%%%% UDP, TCP provides corruption checking using checksum, and multiplexing
%%%%%%%% using ports.  TCP provides the application layer with an abstraction of
%%%%%%%% a transparent end-to-end duplex virtual circuit (byte stream).
%%%%%%%% \cite{rfc1180}
%%%%%%%% 
%%%%%%%% %\subsection{How TCP Operates}
%%%%%%%% %
%%%%%%%% %Similarly to a link-layer Virtual Circuit, there are three phases of
%%%%%%%% %TCP operation:
%%%%%%%% %
%%%%%%%% %\begin{enumerate}
%%%%%%%% %    \item connection set-up
%%%%%%%% %    \item data transfer
%%%%%%%% %    \item connection tear-down
%%%%%%%% %\end{enumerate}
%%%%%%%% %
%%%%%%%% %Connection set-up is performed using a three-way handshake.  Once that
%%%%%%%% %is done, TCP accepts data from the application layer.  The connection
%%%%%%%% %comes to an end when either host terminates the connection, or the
%%%%%%%% %connection times out.
%%%%%%%% %
%%%%%%%% %Every successfully received packet is acknowledged.  Acknowledge
%%%%%%%% %packets can piggy-back on data packets sent to the other host.
%%%%%%%% %
%%%%%%%% %TCP uses a {\em sliding window} to determine whether a given packet has
%%%%%%%% %been received.  At any given time, as many bytes as can fit in the
%%%%%%%% %window can remain unacknowledged.  When the number of bytes goes over
%%%%%%%% %the size of the window, the packet is presumed lost.  The size of the
%%%%%%%% %window changes during the transmission, to accommodate changes in
%%%%%%%% %link throughput, and congestion control.
%%%%%%%% %
%%%%%%%% %%%%%%%%%%%%%%%%%%%%%%%%%%%%%%%%%%%%%%%%%%%%%%%%%%%%%%%%%%%%%%%%%%%%%%%%%%%%%%


%
% Appendices
%

\appendix
\chapter{Result data}

\begin{table}[h]
  \begin{tabular}{ | p{3cm} | p{1cm} | p{8cm} | }
  \hline

  {\em File Name}
  	& {\em Line} 
	& {\em Result}
	\\ \hline
  \hline

  foo.java
      & 3
      & Comment removed --- no effect on semantics
      \\ \hline
  bar.java
      & 42
      & Compile time error: ``bar.java:3: You fool!''
      \\ \hline

  \end{tabular}
  \caption{Line removal}
  \label{remove-line-data}
\end{table} 


Table \ref{remove-line-data} presents the… data!

%
% References
%

%%%%%%%%%%%%%%%%%%%%%%%%%%%%%%%%%%%%%%%%%%%%%%%%%%%%%%%%%%%%%%%%%%%%%%%%%%%%%%
% ``References and presentation (10%) Take care to refer to any source
% material appropriately, using an appropriate citation style (Coxhead 2009 "A
% Referencing Style Guide", http://www.cs.bham.ac.uk/~pxc/refs/refs.html
% [accessed 18 Nov 2010]). The mark of 10% includes appropriate referencing
% and appropriate presentation throughout the document.''
%	-- the assignment

\bibliography{references,rfc}
\thispagestyle{fancy}

\end{document}
