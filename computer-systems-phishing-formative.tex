% computer-systems-phishing-formative.tex -- Report on ‘phishing’
%
% This is a solution of assignment in practicals of the Introduction to
% Networks submodule of the Computer Systems module held at the Durham
% University, Durham, United Kingdom.  Michælmas term 2008.
%
% Copyright © 2008–2009 Jan Minář <rdancer@rdancer.org>
%
% This work is free software; you can redistribute it and/or modify
% it under the terms of the GNU General Public License version 2 (two),
% as published by the Free Software Foundation.
%
% This work is distributed in the hope that it will be useful,
% but WITHOUT ANY WARRANTY; without even the implied warranty of
% MERCHANTABILITY or FITNESS FOR A PARTICULAR PURPOSE.  See the
% GNU General Public License for more details.
%
% You should have received a copy of the GNU General Public License along
% with this work; if not, write to the Free Software Foundation, Inc.,
% 51 Franklin Street, Fifth Floor, Boston, MA 02110-1301 USA.

\documentclass[10pt]{article}


\usepackage[utf8]{inputenc}
\usepackage{graphicx}     
\usepackage{amssymb}    


\author{Jan Minář {\tt <rdancer@rdancer.org>}}
%\date{November 24, 2008}
\title{Computer Systems Practical (Week 7--8)\\Phishing\\Getting cosy amongst the tubes}

\begin{document}

\maketitle

\subsection{What is Phishing?}
``In computing, phishing is a form of social engineering, characterised by attempts to fraudulently acquire sensitive information, such as passwords ...'' \cite{gwavanation:phishing-definition}

\subsection{Short History of Phishing}


\subsection{User Perspective}
\subsection{Attacker Perspective}
\subsection{Application Designer Perspective}
\subsection{Cost Bearer Perspective}
\subsection{Conclusion}

Phishing

% References

\bibliographystyle{prsty}
\bibliography{references}

\end{document}
