% computer-systems-phishing-formative.tex -- Report on ‘phishing’
%
% This is a solution of assignment in practicals of the Introduction to
% Networks submodule of the Computer Systems module held at the Durham
% University, Durham, United Kingdom.  Michælmas term 2008.
%
% Copyright © 2008–2009 Jan Minář <rdancer@rdancer.org>
%
% This work is free software; you can redistribute it and/or modify
% it under the terms of the GNU General Public License version 2 (two),
% as published by the Free Software Foundation.
%
% This work is distributed in the hope that it will be useful,
% but WITHOUT ANY WARRANTY; without even the implied warranty of
% MERCHANTABILITY or FITNESS FOR A PARTICULAR PURPOSE.  See the
% GNU General Public License for more details.
%
% You should have received a copy of the GNU General Public License along
% with this work; if not, write to the Free Software Foundation, Inc.,
% 51 Franklin Street, Fifth Floor, Boston, MA 02110-1301 USA.

\documentclass[10pt]{article}


\usepackage[utf8]{inputenc}
\usepackage{graphicx}     
\usepackage{amssymb}    
\usepackage{harvard}


\author{Jan Minář {\tt <rdancer@rdancer.org>}}
%\date{November 24, 2008}

% No LaTeX command to make a subtitle, but possible using custom code (not
% including due to absence of license, but it is possible):
% <http://groups.google.com/group/comp.text.tex/msg/3aa4f67d8b3a979b?hl=en-EN&pli=1>
\title{Computer Systems Practical (Week 7--8)\\Phishing}

\begin{document}
\bibliographystyle{agsm}

\maketitle

% Note: This is a ‘report’
%
% From the assignment:
%
%    You should include:
%    * A general description of phishing
%    * A description of an instance of phishing
%    * How the network is used and/or effected as a consequence of phishing 
%    * How to prevent further victims of phishing, both a: 
%   	- Technical method; and a
%   	- Non-technical method
%    * Reference section
%

%
% A general description of phishing
%

\section{What is Phishing?}
``In computing, phishing is a form of social engineering, characterised by attempts to fraudulently acquire sensitive information, such as passwords ...'' \cite{gwavanation:phishing-definition}

%
% A description of an instance of phishing 
%

\section{Case Study}

%
% How the network is used and/or effected as a consequence of phishing
%

\section{How the network is used and/or effected as a consequence of phishing}

%
% How to prevent further victims of phishing
%

\section{How to prevent further victims of phishing}

%
% Technical method
%

\subsection{Technical method}

%
% Non-technical method
%

\subsection{Non-technical method}

%
% References
%

\bibliography{references}

\end{document}
