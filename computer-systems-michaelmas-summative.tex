% computer-systems-michaelmas-summative.tex -- Summative report
%
% This is a solution of the summative assignment of the Introduction to
% Networks submodule of the Computer Systems module held at the Durham
% University, Durham, United Kingdom.  Michælmas term 2008.
%
% Copyright © 2008–2009 Jan Minář <rdancer@rdancer.org>
%
% This work is free software; you can redistribute it and/or modify
% it under the terms of the GNU General Public License version 2 (two),
% as published by the Free Software Foundation.
%
% This work is distributed in the hope that it will be useful,
% but WITHOUT ANY WARRANTY; without even the implied warranty of
% MERCHANTABILITY or FITNESS FOR A PARTICULAR PURPOSE.  See the
% GNU General Public License for more details.
%
% You should have received a copy of the GNU General Public License along
% with this work; if not, write to the Free Software Foundation, Inc.,
% 51 Franklin Street, Fifth Floor, Boston, MA 02110-1301 USA.

\documentclass[10pt]{report}


\usepackage[utf8]{inputenc}
\usepackage{graphicx}     
\usepackage{amssymb}    
\usepackage{harvard}
\usepackage{url}


\author{Jan Minář {\tt <rdancer@rdancer.org>}}
%\date{November 24, 2008}

% No LaTeX command to make a subtitle, but possible using custom code (not
% including due to absence of license, but it is possible):
% <http://groups.google.com/group/comp.text.tex/msg/3aa4f67d8b3a979b?hl=en-EN&pli=1>
\title{Computer Systems\\Introduction to Networks\\Summative Assignment}

\begin{document}
\bibliographystyle{agsm}

\maketitle

% Note: This is a ‘report’


%%%%%%%%%%%%%%%%%%%%%%%%%%%%%%%%%%%%%%%%%%%%%%%%%%%%%%%%%%%%%%%%%%%%%%%%%%%%%%
% ``An architecture is a model that allows you to arrange the entities of a
% system into some notion of pattern.  Describe in detail what is meant by ``a
% Client-Server'' architecture.''
%	-- the assignment

\chapter{Client-Server Architecture}
%%%%%%%%%%%%%%%%%%%%%%%%%%%%%%%%%%%%%%%%%%%%%%%%%%%%%%%%%%%%%%%%%%%%%%%%%%%%%%

%\begin{quote}
%``In computing, phishing is a form of social engineering, characterised by
%attempts to fraudulently acquire sensitive information, such as passwords
%[\,\ldots]'' \cite{gwavanation:phishing-definition}
%\end{quote}


%%%%%%%%%%%%%%%%%%%%%%%%%%%%%%%%%%%%%%%%%%%%%%%%%%%%%%%%%%%%%%%%%%%%%%%%%%%%%%
% ``There are TWO types of packet switching networks: virtual circuit and
% datagram.  Describe in detail the key features of each type?''
%	-- the assignment

\chapter{Key Features of Virtual Circuit and Datagram Packet Switching
	Networks}

% Intro

The word ``datagram'' derives from ``telegram'' \cite[p 141]{russell}

% Just plain blabbering from now on

``VC can be built on top of the datagram service'' \cite[p 141]{russell}

VC came from the traditional pre-existing voice telephone network, which was a usually state-wide network, with interstate and overseas links.  On the other hand, packet switching was a cool new technology in the 1970s.  Nowadays, datagram takes over traditional mainstays of VC, however, VC is still being used.  It is possible to use a mixed approach.

Examples of VC: ATM, Frame Relay, X.25
Examples of datagram: internet, GPRS


Three phases of setting up the connection on a VC:

\begin{itemize}
\item[1.] set-up
    \begin{itemize}
    \item Here all the resources are reserved, and everything is set
    \item The way it works, the LCI (Logical Channel Identifier) is the name of the single connection to the next node---some see it as an advantage that instead of using the (long) address, only the (short) LCI is used.  This will save bandwidth and complexity and CPU cycles compared to having to compute the route for every datagram. \cite[p 158]{russell} (the addresses are only used/needed during the initial phase of connection set-up).  However, there are ways to work around this limitation in datagram networks.
    \item Takes time, so the up-front delay can be considerably higher than datagram
    \end{itemize}
\item[2.] data transfer
    \begin{itemize}
    \item nodes use LCI
    \item the latency is considerably negligible, because there are no delays introduced by the management of the link, unlike with datagram networks
    \end{itemize}
\item[3.] tear-down
    \begin{itemize}
        \item the resources are freed
	\item LCI is forgotten
	\item the routing tables are purged
	\item bandwidth/slots are released for use for upcoming VCs
    \end{itemize}
\end{itemize}


\section{History}

In telephony systems, complexity is kept within the network, which allows
``dumb'' end-systems as simple as rotary phones be connected.  Internet was on
the other hand designed to connect sophisticated hosts, and designed
deliberately to be ``as simple as possible''.  Complex functionality is
implemented by the {\em transport} and {\em application} layers.  The internet
network layer is easy to implement over disparate range of {\em link} layer
technologies, such as ``satellite, Ethernet, fiber, or radio''.  It's also easy
to implement new applications, because all that is required is implemented in
the end hosts; changes to the network are not necessary.
\cite[pp349--351]{kurose}

\begin{table}[h]
  \label{vc_dg_comparison}
  \begin{tabular}{ | p{6cm} | p{6cm} | }
  \hline

  Virtual Circuit
  	& Datagram \\ \hline

E.g.\ in telephony systems, complexity is kept within the network, which allows ``dumb'' end-systems as simple as rotary phones be connected \cite[p349]{kurose}
	& Internet was on the other hand designed to connect sophisticated
	hosts, and designed deliberately to be ``as simple as possible''.  Complex functionality is implemented by the {\em transport} and {\em application} layers. \cite[pp349--351]{kurose}
	\\ \hline

maintains a dedicated path through the network between the two end-hosts 
	& routes each packet separately, each packet contains routing
	information header; path can change in-between packets \\ \hline

guarantees delivery, low latency, bandwidth
	& best-effort, with packet loss/duplicity, jitter, data corruption,
	shared bandwidth \\ \hline

its all-or-nothing approach means it will simply fail hard, and tear down the
connection
	&  that when datagram network routes around a malfunctioning router,
	and recovers from errors \\ \hline

naturally maps onto traditional voice networks, which it was developed from and
for
	& in the 1970s newly devised; has not fundamentally changed since, one
	of few technologies of choice for data transmission over long distances
	\cite[p298--299]{stallings} \\ \hline

good for voice \& similar real-time, unfluctuating bandwith requirements, low
latency applications
	& good for applications that transmit data in bursts \\ \hline

upfront delay while the virtual circuit is set up; latency negligible afterwards
	& considerable latency due to intrinsic delays \\ \hline

  \end{tabular} \caption{Comparison of Virtual Circuit and Datagram networks} \end{table} 
Data in Table
% XXX The number in the caption reads 2.1, but here it's just ``2'' -- adding
% the ``.1'' manually
\ref{vc_dg_comparison}.1
are from \cite{kurose} \cite[p298--299]{stallings} \cite{russell}.

The virtues of the shiny VC technology do not prevent errors in other processes and subsystems.  Properly engineered system will function correctly regardless of which of the two approaches is used \cite[p161]{russell}

Some applications can not take advantage of the VC guaranteed sequentiality \cite[p161]{russell}

the VC advantage of link and CPU efficiency due to using the LCI in place of the full address is not so marked when workarounds and real-life costs are taken into account: datagram network does not have to work with the full-length representation of an address, and replacing the (long) address with a (short) LCI does not necessarily prove to result in significant savings \cite[p161]{russell}

\section{Routing in Datagram Networks}

Router maintains mapping route $\rightarrow$ interface

Static routing

Automatic routing protocols, change routing table typically every few minutes. \cite{xxx}  In a VC network, the routing table effectively changes with every VC setup/teardown, i.e.\ constantly.

\section{Hybrid Approach}

It is possible to use a hybrid approach, and in fact many networks currently deployed do so.

VC intereface, datagram network---so that the advantage of having a nice and comfortable interface presented to the host is married with the flexibility and cost-efficiency of a datagram network.





XXX Foo \cite{xxx} \cite{kurose} \cite{russell} \cite{stallings}\\
\cite{slides-lecture-2}
\cite{slides-lecture-6}
\cite{crowcroft}

\section{Conclusion}
%%%%%%%%%%%%%%%%%%%%%%%%%%%%%%%%%%%%%%%%%%%%%%%%%%%%%%%%%%%%%%%%%%%%%%%%%%%%%%



%%%%%%%%%%%%%%%%%%%%%%%%%%%%%%%%%%%%%%%%%%%%%%%%%%%%%%%%%%%%%%%%%%%%%%%%%%%%%%
% ``There are a wide range of application layer protocols including the
% Hypertext Transfer Protocol (HTTP).  Describe HTTP in detail and be sure to
% include in your description HTTP's function (what does it do), design (why
% does it do it), and behaviour (how does it do it).''
%	-- the assignment

\chapter{Hypertext Transfer Protocol}
%%%%%%%%%%%%%%%%%%%%%%%%%%%%%%%%%%%%%%%%%%%%%%%%%%%%%%%%%%%%%%%%%%%%%%%%%%%%%%

\section{Design}
\section{Behaviour}
\section{Function}
\section{Packet Dissection}


%%%%%%%%%%%%%%%%%%%%%%%%%%%%%%%%%%%%%%%%%%%%%%%%%%%%%%%%%%%%%%%%%%%%%%%%%%%%%%
% ``There are TWO types of transport layer service that the Internet provides
% to its applications, connection-oriented services and connectionless
% services.
% ``a) Describe in detail the characteristics of connection-oriented services.
% ``b) Describe in detail the characteristics of connectionless services.''
%	-- the assignment

\chapter{Characteristics of Connection-Oriented and Connectionless Transport
	Layer Services}
%%%%%%%%%%%%%%%%%%%%%%%%%%%%%%%%%%%%%%%%%%%%%%%%%%%%%%%%%%%%%%%%%%%%%%%%%%%%%%



%
% References
%

\bibliography{references}

\end{document}
