% computer-systems-michaelmas-summative.tex -- Summative report
%
% This is a solution of the summative assignment of the Introduction to
% Networks submodule of the Computer Systems module held at the Durham
% University, Durham, United Kingdom.  Michælmas term 2008.
%
% Copyright © 2008–2009 Jan Minář <rdancer@rdancer.org>
%
% This work is free software; you can redistribute it and/or modify
% it under the terms of the GNU General Public License version 2 (two),
% as published by the Free Software Foundation.
%
% This work is distributed in the hope that it will be useful,
% but WITHOUT ANY WARRANTY; without even the implied warranty of
% MERCHANTABILITY or FITNESS FOR A PARTICULAR PURPOSE.  See the
% GNU General Public License for more details.
%
% You should have received a copy of the GNU General Public License along
% with this work; if not, write to the Free Software Foundation, Inc.,
% 51 Franklin Street, Fifth Floor, Boston, MA 02110-1301 USA.

\documentclass[10pt]{report}


\usepackage[utf8]{inputenc}
\usepackage[pdftex]{graphicx}     
\usepackage{amssymb}    
\usepackage{harvard}
\usepackage{url}
\usepackage{fancyhdr}
\usepackage{lastpage}

\pagestyle{fancy}
\fancyhead{}
%\chead{Computer Systems --- Introduction to Networks --- Summative Assignment}
%\chead{Copyright © 2008–2009 Jan Minář {\tt <rdancer@rdancer.org>}}
	%\\ page \thepage\ of \pageref{LastPage}}

\chead{
    Computer Systems --- Introduction to Networks --- Summative Assignment\\
    Copyright © 2008–2009 Jan Minář {\tt <rdancer@rdancer.org>}
}

\author{Jan Minář {\tt <rdancer@rdancer.org>}}
%\date{November 24, 2008}

% No LaTeX command to make a subtitle, but possible using custom code (not
% including due to absence of license, but it is possible):
% <http://groups.google.com/group/comp.text.tex/msg/3aa4f67d8b3a979b?hl=en-EN&pli=1>
\title{Computer Systems\\Introduction to Networks\\Summative Assignment}

\begin{document}
\bibliographystyle{agsm}

\maketitle

% Note: This is a ‘report’


%%%%%%%%%%%%%%%%%%%%%%%%%%%%%%%%%%%%%%%%%%%%%%%%%%%%%%%%%%%%%%%%%%%%%%%%%%%%%%
% ``An architecture is a model that allows you to arrange the entities of a
% system into some notion of pattern.  Describe in detail what is meant by ``a
% Client-Server'' architecture.''
%	-- the assignment

\chapter{Client-Server Architecture}
\thispagestyle{fancy}

In a client-server architecture, the application is distributed over two or
more separate platforms.  The servers offer services which are utilized by the
clients.  One functional unit can act both as a client and a server at the same time
(e.g.\ a web server that is at the same time a client to a database server).
Clients and servers communicate over shared network. \cite[pp3--11]{vaughn}

There is a vast number of applications that follow the client-server model
(most of the ports assigned by IANA correspond to client-server applications
\cite{iana}).  Alternatives to the client-server architecture include
{\em application} architecture and {\em peer-to-peer} architecture.  \cite[p110]{kurose}

There is typically one or a few servers, serving a large number of clients
\cite[p110]{kurose}.  It is not possible for clients to communicate
with each other directly; all communication between clients must be
realized through the server (for example, two Mail User Agents 
can indeed send e-mails to each other, but always via e.g.\ SMTP and IMAP mail servers).

Maintaining a server can be ``infrastructure-intensive'', when the server has
to withstand very many requests.  Often a server farm is deployed, so that the
load is shared between multiple machines.  \cite[p110]{kurose}

\section{Transport Layer View}

The client initiates the communication by sending a request to the
server.  The server only {\em responds}, and can never {\em initiate}
communication.  If the initial request is to succeed, the server process
must have had requested the operating system to listen on a given (TCP
or UDP) port.  The port numbers and their corresponding services are
maintained by a central registry \cite{iana}, so that for example a POP3
client will be able to connect to a POP3 server just by knowing the
server IP address.  The port number only has to be specified when it differs
from the default.  Once the client sends in a request to the correct port and
address, request is received by the operating system of the server machine, the
operating system passes the request on to the server process.  The server
process responds to the request, and two-way communication ensues.

% % % % % % % % % % % % % % % % % % % % % % % % % % % % % % % % % % % % % % %

\section{Thick and Thin Clients}

A thin client system does most of the data processing on the server side.  This
allows the client to be relatively low-powered, which can mean lower costs
(network terminals used instead of full-blown PCs, with the applications running
on an application server), or perhaps longer battery life (mobile phone or a PDA
running a video-processing application client, with the actual computationally
intensive video re-encoding performed on the server accessed over the mobile network).  Early web browsers were thin clients.

A thick client does most of the data processing itself.  This approach
does not suffer from the limitations of the network, such as latency.
For example, high quality video playback is a bandwidth-intensive
application, and it is often more practical to transport the video over
the network in a compressed form, and have the client do the
computationally intensive decompression, thus saving bandwidth.
Contemporary web browsers are thick clients.

%%%%%%%%%%%%%%%%%%%%%%%%%%%%%%%%%%%%%%%%%%%%%%%%%%%%%%%%%%%%%%%%%%%%%%%%%%%%%%

%%%%%%%%%%%%%%%%%%%%%%%%%%%%%%%%%%%%%%%%%%%%%%%%%%%%%%%%%%%%%%%%%%%%%%%%%%%%%%
% ``There are TWO types of packet switching networks: virtual circuit and
% datagram.  Describe in detail the key features of each type?''
%	-- the assignment

\chapter{Key Features of Virtual Circuit and Datagram Packet Switching Networks}
\thispagestyle{fancy}

% Intro


% Just plain blabbering from now on

VC came from the traditional pre-existing voice telephone network, which
was usually a state-wide network, with interstate and overseas links.
The word ``datagram'' itself derives from ``telegram'' \cite[p141]{russell}.  On
the other hand, packet switching was developed in the 1970s as means of
efficient transmission of data over long distances.  Nowadays, datagram takes over traditional mainstays of VC, however, VC is still being used.  It is possible to use a mixed approach.

There are three phases of setting up the connection on a VC:

\begin{enumerate}
\item set-up
    \begin{itemize}
    %\item LCI (Logical Channel Identifier; the name of the single connection to the next node) is chosen
    \item the path through the network is determined
    \item resources such as bandwidth/time slot are reserved, and everything is set
    \item this can take some amount of time, creating a set-up delay
    \end{itemize}
\item data transfer
    \begin{itemize}
    %\item nodes use LCI
    \item the path does not changed for the duration of the connection
    \item resources remain reserved even when not actually needed
    \item the latency is negligible, because there are no delays introduced by the management of the link, unlike with datagram networks
    \end{itemize}
\item tear-down
    \begin{itemize}
        %\item the resources are freed
	%\item LCI is forgotten
	\item the routing table entries are purged
	\item bandwidth/slots are released for use for future VCs
    \end{itemize}
\end{enumerate}

Datagram networks route each packet individually.

\section{History}

In telephony systems, complexity is kept within the network, which allows
``dumb'' end-systems as simple as rotary phones be connected.  Internet was on
the other hand designed to connect sophisticated hosts, and designed
deliberately to be ``as simple as possible''.  Complex functionality is
implemented by the {\em transport} and {\em application} layers.  The internet
{\em network} layer is easy to implement over disparate range of {\em link} layer
technologies, such as ``satellite, Ethernet, fiber, or radio''.  It's also easy
to implement new applications, because all that is required is to implement them
in the end hosts; changes to the network are not necessary.
\cite[pp349--351]{kurose}

\section{Comparison}

\begin{table}[h]
  \begin{tabular}{ | p{6cm} | p{6cm} | }
  \hline

  {\em Virtual Circuit}
  	& {\em Datagram} \\ \hline
  \hline

E.g.\ in telephony systems, complexity is kept within the network, which allows ``dumb'' end-systems as simple as rotary phones be connected \cite[p349]{kurose}
	& Internet was on the other hand designed to connect sophisticated
	hosts, and designed deliberately to be ``as simple as possible''.  Complex functionality is implemented by the {\em transport} and {\em application} layers. \cite[pp349--351]{kurose}
	\\ \hline

maintains a dedicated path through the network between the two end-hosts 
	& routes each packet separately, each packet contains routing
	information header; path can change in-between packets \\ \hline

guarantees delivery, low latency, bandwidth
	& best-effort, with packet loss/duplicity, jitter, data corruption,
	shared bandwidth \\ \hline

its all-or-nothing approach means it will simply fail hard, and tear down the
connection
	&  that when datagram network routes around a malfunctioning router,
	and recovers from errors \\ \hline

naturally maps onto traditional voice networks, which it was developed from and
for
	& in the 1970s newly devised; has not fundamentally changed since, one
	of few technologies of choice for data transmission over long distances
	\cite[p298--299]{stallings} \\ \hline

good for voice \& similar real-time, low-latency applications that require
	approximately the same amount of bandwidth over time
	& good for applications that transmit data in bursts \\ \hline

upfront delay while the virtual circuit is set up; latency negligible afterwards
	& considerable latency due to intrinsic delays \\ \hline

  \end{tabular}
  \caption{Comparison of Virtual Circuit and Datagram Networks}
  \label{vcdgcomparison}

\end{table} 


Table \ref{vcdgcomparison} compiled from data in \cite{kurose},
\cite[p298--299]{stallings} and \cite{russell}.

Some see it as an advantage that instead of using the (long) address,
only the (short) LCI is used.  This will save bandwidth, complexity and
processing power, compared to having to compute the route for every
datagram. \cite[p158]{russell} (the addresses are only used/needed
during the initial phase of connection set-up).  However, there are ways
to approximate these advantages even in datagram networks.

The virtues of the VC technology do not prevent errors in other parts of
an application.  Properly engineered system will function correctly
regardless of which of the two approaches is used \cite[p161]{russell}

Some applications can not take advantage of the VC guaranteed
sequentiality \cite[p161]{russell}

The VC advantage of link and CPU efficiency due to using the LCI in
place of the full address is not so marked when workarounds and
real-life costs are taken into account: datagram network does not have
to work with the full-length representation of an address, and replacing
the (long) address with a (short) LCI does not necessarily prove to
result in significant savings \cite[p161]{russell}

\section{Routing in Datagram Networks}

When a datagram router receives a packet, it needs to decide which next
hop it should send it to, i.e.\ which interface to forward it via.  It
would be impractical to store this information for every possible
destination address, and therefore the routers mostly store only
aggregate routes (multiple adjacent addresses represented by a common
prefix).  Often an address is
comprised of a prefix, and a host part.  The routes are then
decided with respect to the network prefixes, not the individual
addresses.  A router can maintain two different routes for two network
prefixes in such a way that one prefix is contained in the other one.
In that case, the longer prefix takes precedence.  It is said that the
corresponding route is more specific.

As using hard-coded, or static routing would not be feasible for busy
routers with many connections, automatic routing protocols have been
devised that change routing table typically every few minutes.  In
contrast to that, in a VC network, the routing table constantly changes
with every VC setup/tear-down, many times a second.

\section{Hybrid Approach}

``VC can be built on top of the datagram service'' \cite[p141]{russell}

Many currently deployed networks behave like a VC towards the end-hosts,
and are really internally working as datagram networks---the advantage
of having a comfortable interface presented to the host is married with
the flexibility and cost-efficiency of a datagram network.
%%%%%%%%%%%%%%%%%%%%%%%%%%%%%%%%%%%%%%%%%%%%%%%%%%%%%%%%%%%%%%%%%%%%%%%%%%%%%%



%%%%%%%%%%%%%%%%%%%%%%%%%%%%%%%%%%%%%%%%%%%%%%%%%%%%%%%%%%%%%%%%%%%%%%%%%%%%%%
% ``There are a wide range of application layer protocols including the
% Hypertext Transfer Protocol (HTTP).  Describe HTTP in detail and be sure to
% include in your description HTTP's function (what does it do), design (why
% does it do it), and behaviour (how does it do it).''
%	-- the assignment

\chapter{Hypertext Transfer Protocol}
\thispagestyle{fancy}
%%%%%%%%%%%%%%%%%%%%%%%%%%%%%%%%%%%%%%%%%%%%%%%%%%%%%%%%%%%%%%%%%%%%%%%%%%%%%%

The Hypertext Transfer Protocol is a non-proprietary client-server
stateless application layer protocol that communicates over TCP.
\cite{rfc1945} \cite{rfc2616}.
\cite[pp122--124]{kurose}

HTTP is the most often used data transport application protocol on the
World Wide Web.

\section{Design}

HTTP uses reliable transport provided by TCP.  The protocol leverages
the very powerful concept of URIs/URLs \cite{rfc1738}.  In version 1.1,
several improvements were introduced that improve the performance of
HTTP.  The specification was originally about 60 pages long, but it is
now close to three times that, covering various issues. \cite{rfc2616}.

\section{Behaviour}

% XXX

\section{Function}

The main function of HTTP is to retrieve and manipulate objects denoted
by Uniform Resource Locators (URLs).  URL is conceptually a subset of
Uniform Resource Identifier---URI can denote things that are not
retrievable, such as a book, telephone number, an e-mail address, or for
example a program that can be queried via HTTP.  URL on the other hand
denotes intrinsically retrievable objects. \cite{rfc1738}

HTTP has auxiliary capabilities represented by the methods HEAD, PUT,
DELETE, OPTIONS, TRACE and CONNECT.

\section{Packet Dissection}

We have prepared two HTTP packets out of a TCP stream that was the
result of retrieving a single document from a remote WWW server.  The
upper half of the images shows the interpretation, the lower half then
the on-the-wire data in hexadecimal and ASCII representation.  We used
{\em Wireshark} 1.0.0 to perform the packet analysis.

\begin{figure}[p]
    \label{httprequest}
    \centering
    % Border around the image
    \setlength\fboxsep{0pt}
    \setlength\fboxrule{0.5pt}
    \fbox{
	\includegraphics[width=0.8\textwidth]{http-get.png}
    }
    \caption{HTTP request}
\end{figure}

Figure
%
% XXX
%\refttprequest}
3.1
%
shows the HTTP request.  TCP packet sent to port {\em
http} (80), the well-known port number assigned to HTTP.  The {\em Host}
header is sent to allow for multiple virtual servers on the same IP
address.  Keep-Alive is set despite the fact that the User Agent is
going to download just this one page---in case of a redirect.

\begin{figure}[p]
    \label{httpresponse}
    \centering
    % Border around the image
    \setlength\fboxsep{0pt}
    \setlength\fboxrule{0.5pt}
    \fbox{
	\includegraphics[width=0.8\textwidth]{http-200-ok.png}
    }
    \caption{HTTP response}
\end{figure}

Figure
%
% XXX
%\ref{httpresponse}
3.2
%
shows the HTTP response generated by the server at
example.com in reaction to our request.  The number of headers is
considerably larger.  Unlike our request, the response utilizes HTTP
version 1.1.

%%%%%%%%%%%%%%%%%%%%%%%%%%%%%%%%%%%%%%%%%%%%%%%%%%%%%%%%%%%%%%%%%%%%%%%%%%%%%%
% ``There are TWO types of transport layer service that the Internet provides
% to its applications, connection-oriented services and connectionless
% services.
% ``a) Describe in detail the characteristics of connection-oriented services.
% ``b) Describe in detail the characteristics of connectionless services.''
%	-- the assignment

\chapter{Characteristics of Connection-Oriented and Connectionless Transport
	Layer Services}
\thispagestyle{fancy}

The two TCP/IP protocol suit transport layer protocols are UDP and TCP.
We will show the strengths and weaknesses, and describe briefly how
each of these two works.

\section{User Datagram Protocol}

UDP is defined in \cite{rfc768}.  It is the datagram transport in the
TCP/IP (i.e.\ internet) protocol suite.  It is a very thin layer on top
of IP, as it only provides a checksum and port number (enabling
multiplexing, i.e.\ having more than one connection from one host to
another via UDP). (Cf.\ Fig.\ \ref{udpheader}) \cite{rfc1180}

\begin{figure}[h]
    \label{udpheader}
    \centering
    % Border around the image
	\begin{verbatim}
              0      7 8     15 16    23 24    31  
              +--------+--------+--------+--------+
              |     Source      |   Destination   |
              |      Port       |      Port       |
              +--------+--------+--------+--------+
              |                 |                 |
              |     Length      |    Checksum     |
              +--------+--------+--------+--------+
              |                                    
              |          data octets ...           
              +---------------- ...                
	\end{verbatim}
    \caption{User Datagram Protocol Header
	    % XXX DNW
	    %\cite{rfc768}
	    (Postel 1980, p1) % Verbatim: DWIT!
    }
\end{figure}

The connectionless service only provides the bare minimum necessary for
a transport layer to provide.  This can be a positive feature, because
it means low overhead and low implementation complexity.

UDP provides the application layer with the ability to send and receive
short fixed-length datagrams.  If this is not enough, the application
must implement the missing features in a higher layer, or use another
transport layer protocol, such as TCP.

\section{Transport Control Protocol}

TCP \cite{rfc793} is the most commonly used transport protocol in the
TCP/IP suite.  In order to provide a connection-oriented service, the
protocol is necessarily slightly more complex than its connectionless
counterpart.  The complexity comes at cost in bandwidth and processing
overhead.

TCP provides in-order reliable delivery (with automatic retransmission
of lost packets) and automatic congestion sensing and adaptation.  Like
UDP, TCP provides corruption checking using checksum, and multiplexing
using ports.  TCP provides the application layer with an abstraction of
a transparent end-to-end duplex virtual circuit (byte stream).
\cite{rfc1180}

\subsection{How TCP Operates}

Similarly to a link-layer Virtual Circuit, there are three phases of
TCP operation:

\begin{enumerate}
    \item connection set-up
    \item data transfer
    \item connection tear-down
\end{enumerate}

Connection set-up is performed using a three-way handshake.  Once that
is done, TCP accepts data from the application layer.  The connection
comes to an end when either host terminates the connection, or the
connection may time out.

Every successfully received packet must be acknowledged.  Acknowledge
packets can piggy-back on data packets sent to the other host.

TCP uses a {\em sliding window} to determine whether a given packet has
been received.  At any given time, as many bytes as can fit in the
window can remain unacknowledged.  When the number of bytes goes over
the size of the window, the packet is presumed lost.  The size of the
window changes during the transmission, to accommodate changes in
link throughput, and congestion control.

%%%%%%%%%%%%%%%%%%%%%%%%%%%%%%%%%%%%%%%%%%%%%%%%%%%%%%%%%%%%%%%%%%%%%%%%%%%%%%



%
% References
%

\bibliography{references}
\thispagestyle{fancy}

\end{document}
