% computer-systems-michaelmas-summative.tex -- Summative report
%
% This is a solution of the summative assignment of the Introduction to
% Networks submodule of the Computer Systems module held at the Durham
% University, Durham, United Kingdom.  Michælmas term 2008.
%
% Copyright © 2008–2009 Jan Minář <rdancer@rdancer.org>
%
% This work is free software; you can redistribute it and/or modify
% it under the terms of the GNU General Public License version 2 (two),
% as published by the Free Software Foundation.
%
% This work is distributed in the hope that it will be useful,
% but WITHOUT ANY WARRANTY; without even the implied warranty of
% MERCHANTABILITY or FITNESS FOR A PARTICULAR PURPOSE.  See the
% GNU General Public License for more details.
%
% You should have received a copy of the GNU General Public License along
% with this work; if not, write to the Free Software Foundation, Inc.,
% 51 Franklin Street, Fifth Floor, Boston, MA 02110-1301 USA.

\documentclass[10pt]{report}


\usepackage[utf8]{inputenc}
\usepackage{graphicx}     
\usepackage{amssymb}    
\usepackage{harvard}
\usepackage{url}


\author{Jan Minář {\tt <rdancer@rdancer.org>}}
%\date{November 24, 2008}

% No LaTeX command to make a subtitle, but possible using custom code (not
% including due to absence of license, but it is possible):
% <http://groups.google.com/group/comp.text.tex/msg/3aa4f67d8b3a979b?hl=en-EN&pli=1>
\title{Computer Systems\\Introduction to Networks\\Summative Assignment}

\begin{document}
\bibliographystyle{agsm}

\maketitle

% Note: This is a ‘report’


%%%%%%%%%%%%%%%%%%%%%%%%%%%%%%%%%%%%%%%%%%%%%%%%%%%%%%%%%%%%%%%%%%%%%%%%%%%%%%
% ``An architecture is a model that allows you to arrange the entities of a
% system into some notion of pattern.  Describe in detail what is meant by ``a
% Client-Server'' architecture.''
%	-- the assignment

\chapter{Client-Server Architecture}
%%%%%%%%%%%%%%%%%%%%%%%%%%%%%%%%%%%%%%%%%%%%%%%%%%%%%%%%%%%%%%%%%%%%%%%%%%%%%%

%\begin{quote}
%``In computing, phishing is a form of social engineering, characterised by
%attempts to fraudulently acquire sensitive information, such as passwords
%[\,\ldots]'' \cite{gwavanation:phishing-definition}
%\end{quote}


%%%%%%%%%%%%%%%%%%%%%%%%%%%%%%%%%%%%%%%%%%%%%%%%%%%%%%%%%%%%%%%%%%%%%%%%%%%%%%
% ``There are TWO types of packet switching networks: virtual circuit and
% datagram.  Describe in detail the key features of each type?''
%	-- the assignment

\chapter{Key Features of Virtual Circuit and Datagram Packet Switching
	Networks}


Foo \cite{xxx} \cite{kurose} \cite{russell} \cite{stallings}

\section{Conclusion}
%%%%%%%%%%%%%%%%%%%%%%%%%%%%%%%%%%%%%%%%%%%%%%%%%%%%%%%%%%%%%%%%%%%%%%%%%%%%%%



%%%%%%%%%%%%%%%%%%%%%%%%%%%%%%%%%%%%%%%%%%%%%%%%%%%%%%%%%%%%%%%%%%%%%%%%%%%%%%
% ``There are a wide range of application layer protocols including the
% Hypertext Transfer Protocol (HTTP).  Describe HTTP in detail and be sure to
% include in your description HTTP's function (what does it do), design (why
% does it do it), and behaviour (how does it do it).''
%	-- the assignment

\chapter{Hypertext Transfer Protocol}
%%%%%%%%%%%%%%%%%%%%%%%%%%%%%%%%%%%%%%%%%%%%%%%%%%%%%%%%%%%%%%%%%%%%%%%%%%%%%%

\section{Design}
\section{Behaviour}
\section{Function}
\section{Packet Dissection}


%%%%%%%%%%%%%%%%%%%%%%%%%%%%%%%%%%%%%%%%%%%%%%%%%%%%%%%%%%%%%%%%%%%%%%%%%%%%%%
% ``There are TWO types of transport layer service that the Internet provides
% to its applications, connection-oriented services and connectionless
% services.
% ``a) Describe in detail the characteristics of connection-oriented services.
% ``b) Describe in detail the characteristics of connectionless services.''
%	-- the assignment

\chapter{Characteristics of Connection-Oriented and Connectionless Transport
	Layer Services}
%%%%%%%%%%%%%%%%%%%%%%%%%%%%%%%%%%%%%%%%%%%%%%%%%%%%%%%%%%%%%%%%%%%%%%%%%%%%%%



%
% References
%

\bibliography{references}

\end{document}
